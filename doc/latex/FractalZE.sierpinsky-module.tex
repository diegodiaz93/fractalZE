%
% API Documentation for API Documentation
% Module FractalZE.sierpinsky
%
% Generated by epydoc 3.0.1
% [Fri Nov 28 18:59:36 2014]
%

%%%%%%%%%%%%%%%%%%%%%%%%%%%%%%%%%%%%%%%%%%%%%%%%%%%%%%%%%%%%%%%%%%%%%%%%%%%
%%                          Module Description                           %%
%%%%%%%%%%%%%%%%%%%%%%%%%%%%%%%%%%%%%%%%%%%%%%%%%%%%%%%%%%%%%%%%%%%%%%%%%%%

    \index{FractalZE \textit{(package)}!FractalZE.sierpinsky \textit{(module)}|(}
\section{Module FractalZE.sierpinsky}

    \label{FractalZE:sierpinsky}

%%%%%%%%%%%%%%%%%%%%%%%%%%%%%%%%%%%%%%%%%%%%%%%%%%%%%%%%%%%%%%%%%%%%%%%%%%%
%%                               Variables                               %%
%%%%%%%%%%%%%%%%%%%%%%%%%%%%%%%%%%%%%%%%%%%%%%%%%%%%%%%%%%%%%%%%%%%%%%%%%%%

  \subsection{Variables}

    \vspace{-1cm}
\hspace{\varindent}\begin{longtable}{|p{\varnamewidth}|p{\vardescrwidth}|l}
\cline{1-2}
\cline{1-2} \centering \textbf{Name} & \centering \textbf{Description}& \\
\cline{1-2}
\endhead\cline{1-2}\multicolumn{3}{r}{\small\textit{continued on next page}}\\\endfoot\cline{1-2}
\endlastfoot\raggedright \_\-\_\-p\-a\-c\-k\-a\-g\-e\-\_\-\_\- & \raggedright \textbf{Value:} 
{\tt \texttt{'}\texttt{FractalZE}\texttt{'}}&\\
\cline{1-2}
\end{longtable}


%%%%%%%%%%%%%%%%%%%%%%%%%%%%%%%%%%%%%%%%%%%%%%%%%%%%%%%%%%%%%%%%%%%%%%%%%%%
%%                           Class Description                           %%
%%%%%%%%%%%%%%%%%%%%%%%%%%%%%%%%%%%%%%%%%%%%%%%%%%%%%%%%%%%%%%%%%%%%%%%%%%%

    \index{FractalZE \textit{(package)}!FractalZE.sierpinsky \textit{(module)}!FractalZE.sierpinsky.Sierpinsky \textit{(class)}|(}
\subsection{Class Sierpinsky}

    \label{FractalZE:sierpinsky:Sierpinsky}
\begin{tabular}{cccccc}
% Line for FractalZE.fractal.Fractal, linespec=[False]
\multicolumn{2}{r}{\settowidth{\BCL}{FractalZE.fractal.Fractal}\multirow{2}{\BCL}{FractalZE.fractal.Fractal}}
&&
  \\\cline{3-3}
  &&\multicolumn{1}{c|}{}
&&
  \\
&&\multicolumn{2}{l}{\textbf{FractalZE.sierpinsky.Sierpinsky}}
\end{tabular}

\textbf{Known Subclasses:} FractalZE.sierpinskyRectangular.SierpinskyRectangular

Fractal o Triangulo de Sierpinsky. Construccion:

\begin{enumerate}

\setlength{\parskip}{0.5ex}
  \item Se parte de un triangulo equilatero.

  \item Se divide cada lado en dos obteniendose 3 puntos.

  \item Al unir los 3 puntos, el triangulo queda dividido en 4 triangulos.

  \item Se descarta el del medio, es decir el que esta invertido.

  \item Se repite esto sucesivamente con cada triangulo.

\end{enumerate}


%%%%%%%%%%%%%%%%%%%%%%%%%%%%%%%%%%%%%%%%%%%%%%%%%%%%%%%%%%%%%%%%%%%%%%%%%%%
%%                                Methods                                %%
%%%%%%%%%%%%%%%%%%%%%%%%%%%%%%%%%%%%%%%%%%%%%%%%%%%%%%%%%%%%%%%%%%%%%%%%%%%

  \subsubsection{Methods}

    \label{FractalZE:sierpinsky:Sierpinsky:__init__}
    \index{FractalZE \textit{(package)}!FractalZE.sierpinsky \textit{(module)}!FractalZE.sierpinsky.Sierpinsky \textit{(class)}!FractalZE.sierpinsky.Sierpinsky.\_\_init\_\_ \textit{(method)}}

    \vspace{0.5ex}

\hspace{.8\funcindent}\begin{boxedminipage}{\funcwidth}

    \raggedright \textbf{\_\_init\_\_}(\textit{self}, \textit{l}={\tt 1})

    \vspace{-1.5ex}

    \rule{\textwidth}{0.5\fboxrule}
\setlength{\parskip}{2ex}
\setlength{\parskip}{1ex}
      \textbf{Parameters}
      \vspace{-1ex}

      \begin{quote}
        \begin{Ventry}{x}

          \item[l]

          longitud del lado maximo(cuando n=0).default = 1.

            {\it (type=float)}

        \end{Ventry}

      \end{quote}

    \end{boxedminipage}

    \label{FractalZE:sierpinsky:Sierpinsky:countTriangles}
    \index{FractalZE \textit{(package)}!FractalZE.sierpinsky \textit{(module)}!FractalZE.sierpinsky.Sierpinsky \textit{(class)}!FractalZE.sierpinsky.Sierpinsky.countTriangles \textit{(method)}}

    \vspace{0.5ex}

\hspace{.8\funcindent}\begin{boxedminipage}{\funcwidth}

    \raggedright \textbf{countTriangles}(\textit{self}, \textit{n})

    \vspace{-1.5ex}

    \rule{\textwidth}{0.5\fboxrule}
\setlength{\parskip}{2ex}
    Devuelve la cantidad de triangulos del termino n.

\setlength{\parskip}{1ex}
      \textbf{Parameters}
      \vspace{-1ex}

      \begin{quote}
        \begin{Ventry}{x}

          \item[n]

          numero de termino de la sucesion

            {\it (type=int)}

        \end{Ventry}

      \end{quote}

      \textbf{Return Value}
    \vspace{-1ex}

      \begin{quote}
      cantidad de triangulos.

      {\it (type=int)}

      \end{quote}

    \end{boxedminipage}

    \label{FractalZE:sierpinsky:Sierpinsky:triangleArea}
    \index{FractalZE \textit{(package)}!FractalZE.sierpinsky \textit{(module)}!FractalZE.sierpinsky.Sierpinsky \textit{(class)}!FractalZE.sierpinsky.Sierpinsky.triangleArea \textit{(method)}}

    \vspace{0.5ex}

\hspace{.8\funcindent}\begin{boxedminipage}{\funcwidth}

    \raggedright \textbf{triangleArea}(\textit{self}, \textit{n})

    \vspace{-1.5ex}

    \rule{\textwidth}{0.5\fboxrule}
\setlength{\parskip}{2ex}
    Devuelve el area ocupada por un solo triangulo del termino n.

\setlength{\parskip}{1ex}
      \textbf{Parameters}
      \vspace{-1ex}

      \begin{quote}
        \begin{Ventry}{x}

          \item[n]

          numero de termino de la sucesion

            {\it (type=int)}

        \end{Ventry}

      \end{quote}

      \textbf{Return Value}
    \vspace{-1ex}

      \begin{quote}
      area ocupada por un triangulo.

      {\it (type=float)}

      \end{quote}

    \end{boxedminipage}

    \label{FractalZE:sierpinsky:Sierpinsky:totalArea}
    \index{FractalZE \textit{(package)}!FractalZE.sierpinsky \textit{(module)}!FractalZE.sierpinsky.Sierpinsky \textit{(class)}!FractalZE.sierpinsky.Sierpinsky.totalArea \textit{(method)}}

    \vspace{0.5ex}

\hspace{.8\funcindent}\begin{boxedminipage}{\funcwidth}

    \raggedright \textbf{totalArea}(\textit{self}, \textit{n})

    \vspace{-1.5ex}

    \rule{\textwidth}{0.5\fboxrule}
\setlength{\parskip}{2ex}
    Devuelve el area total.

\setlength{\parskip}{1ex}
      \textbf{Parameters}
      \vspace{-1ex}

      \begin{quote}
        \begin{Ventry}{x}

          \item[n]

          numero de termino de la sucesion

            {\it (type=int)}

        \end{Ventry}

      \end{quote}

      \textbf{Return Value}
    \vspace{-1ex}

      \begin{quote}
      area total.

      {\it (type=float)}

      \end{quote}

    \end{boxedminipage}

    \label{FractalZE:sierpinsky:Sierpinsky:trianglePerimeter}
    \index{FractalZE \textit{(package)}!FractalZE.sierpinsky \textit{(module)}!FractalZE.sierpinsky.Sierpinsky \textit{(class)}!FractalZE.sierpinsky.Sierpinsky.trianglePerimeter \textit{(method)}}

    \vspace{0.5ex}

\hspace{.8\funcindent}\begin{boxedminipage}{\funcwidth}

    \raggedright \textbf{trianglePerimeter}(\textit{self}, \textit{n})

    \vspace{-1.5ex}

    \rule{\textwidth}{0.5\fboxrule}
\setlength{\parskip}{2ex}
    Devuelve el perimetro de un triangulo del termino n.

\setlength{\parskip}{1ex}
      \textbf{Parameters}
      \vspace{-1ex}

      \begin{quote}
        \begin{Ventry}{x}

          \item[n]

          numero de termino de la sucesion

            {\it (type=int)}

        \end{Ventry}

      \end{quote}

      \textbf{Return Value}
    \vspace{-1ex}

      \begin{quote}
      perimetro de un triangulo.

      {\it (type=float)}

      \end{quote}

    \end{boxedminipage}

    \label{FractalZE:sierpinsky:Sierpinsky:totalPerimeter}
    \index{FractalZE \textit{(package)}!FractalZE.sierpinsky \textit{(module)}!FractalZE.sierpinsky.Sierpinsky \textit{(class)}!FractalZE.sierpinsky.Sierpinsky.totalPerimeter \textit{(method)}}

    \vspace{0.5ex}

\hspace{.8\funcindent}\begin{boxedminipage}{\funcwidth}

    \raggedright \textbf{totalPerimeter}(\textit{self}, \textit{n})

    \vspace{-1.5ex}

    \rule{\textwidth}{0.5\fboxrule}
\setlength{\parskip}{2ex}
    Devuelve la suma de los perimetros de todos los triangulos del termino 
    n.

\setlength{\parskip}{1ex}
      \textbf{Parameters}
      \vspace{-1ex}

      \begin{quote}
        \begin{Ventry}{x}

          \item[n]

          numero de termino de la sucesion

            {\it (type=int)}

        \end{Ventry}

      \end{quote}

      \textbf{Return Value}
    \vspace{-1ex}

      \begin{quote}
      perimetro total.

      {\it (type=float)}

      \end{quote}

    \end{boxedminipage}

    \label{FractalZE:sierpinsky:Sierpinsky:triangleHeight}
    \index{FractalZE \textit{(package)}!FractalZE.sierpinsky \textit{(module)}!FractalZE.sierpinsky.Sierpinsky \textit{(class)}!FractalZE.sierpinsky.Sierpinsky.triangleHeight \textit{(method)}}

    \vspace{0.5ex}

\hspace{.8\funcindent}\begin{boxedminipage}{\funcwidth}

    \raggedright \textbf{triangleHeight}(\textit{self}, \textit{n})

    \vspace{-1.5ex}

    \rule{\textwidth}{0.5\fboxrule}
\setlength{\parskip}{2ex}
    Devuelve la altura de un triangulo del termino n.

\setlength{\parskip}{1ex}
      \textbf{Parameters}
      \vspace{-1ex}

      \begin{quote}
        \begin{Ventry}{x}

          \item[n]

          numero de termino de la sucesion

            {\it (type=int)}

        \end{Ventry}

      \end{quote}

      \textbf{Return Value}
    \vspace{-1ex}

      \begin{quote}
      altura de un triangulo.

      {\it (type=float)}

      \end{quote}

    \end{boxedminipage}

    \label{FractalZE:sierpinsky:Sierpinsky:triangleWidth}
    \index{FractalZE \textit{(package)}!FractalZE.sierpinsky \textit{(module)}!FractalZE.sierpinsky.Sierpinsky \textit{(class)}!FractalZE.sierpinsky.Sierpinsky.triangleWidth \textit{(method)}}

    \vspace{0.5ex}

\hspace{.8\funcindent}\begin{boxedminipage}{\funcwidth}

    \raggedright \textbf{triangleWidth}(\textit{self}, \textit{n})

    \vspace{-1.5ex}

    \rule{\textwidth}{0.5\fboxrule}
\setlength{\parskip}{2ex}
    Devuelve el ancho de un triangulo del termino n.

\setlength{\parskip}{1ex}
      \textbf{Parameters}
      \vspace{-1ex}

      \begin{quote}
        \begin{Ventry}{x}

          \item[n]

          numero de termino de la sucesion

            {\it (type=int)}

        \end{Ventry}

      \end{quote}

      \textbf{Return Value}
    \vspace{-1ex}

      \begin{quote}
      ancho de un triangulo.

      {\it (type=float)}

      \end{quote}

    \end{boxedminipage}

    \label{FractalZE:sierpinsky:Sierpinsky:getWidth}
    \index{FractalZE \textit{(package)}!FractalZE.sierpinsky \textit{(module)}!FractalZE.sierpinsky.Sierpinsky \textit{(class)}!FractalZE.sierpinsky.Sierpinsky.getWidth \textit{(method)}}

    \vspace{0.5ex}

\hspace{.8\funcindent}\begin{boxedminipage}{\funcwidth}

    \raggedright \textbf{getWidth}(\textit{self}, \textit{n}={\tt 0})

    \vspace{-1.5ex}

    \rule{\textwidth}{0.5\fboxrule}
\setlength{\parskip}{2ex}
    Devuelve el ancho del fractal. Siempre es igual a l

\setlength{\parskip}{1ex}
      \textbf{Return Value}
    \vspace{-1ex}

      \begin{quote}
      ancho del fractal

      {\it (type=float)}

      \end{quote}

    \end{boxedminipage}

    \label{FractalZE:sierpinsky:Sierpinsky:getHeight}
    \index{FractalZE \textit{(package)}!FractalZE.sierpinsky \textit{(module)}!FractalZE.sierpinsky.Sierpinsky \textit{(class)}!FractalZE.sierpinsky.Sierpinsky.getHeight \textit{(method)}}

    \vspace{0.5ex}

\hspace{.8\funcindent}\begin{boxedminipage}{\funcwidth}

    \raggedright \textbf{getHeight}(\textit{self}, \textit{n}={\tt 0})

    \vspace{-1.5ex}

    \rule{\textwidth}{0.5\fboxrule}
\setlength{\parskip}{2ex}
    Devuelve la altura del fractal.

\setlength{\parskip}{1ex}
      \textbf{Return Value}
    \vspace{-1ex}

      \begin{quote}
      ancho del fractal

      {\it (type=float)}

      \end{quote}

    \end{boxedminipage}


\large{\textbf{\textit{Inherited from FractalZE.fractal.Fractal\textit{(Section \ref{FractalZE:fractal:Fractal})}}}}

\begin{quote}
graph()
\end{quote}
    \index{FractalZE \textit{(package)}!FractalZE.sierpinsky \textit{(module)}!FractalZE.sierpinsky.Sierpinsky \textit{(class)}|)}
    \index{FractalZE \textit{(package)}!FractalZE.sierpinsky \textit{(module)}|)}
