%
% API Documentation for API Documentation
% Module FractalZE.pascal
%
% Generated by epydoc 3.0.1
% [Fri Nov 28 18:59:36 2014]
%

%%%%%%%%%%%%%%%%%%%%%%%%%%%%%%%%%%%%%%%%%%%%%%%%%%%%%%%%%%%%%%%%%%%%%%%%%%%
%%                          Module Description                           %%
%%%%%%%%%%%%%%%%%%%%%%%%%%%%%%%%%%%%%%%%%%%%%%%%%%%%%%%%%%%%%%%%%%%%%%%%%%%

    \index{FractalZE \textit{(package)}!FractalZE.pascal \textit{(module)}|(}
\section{Module FractalZE.pascal}

    \label{FractalZE:pascal}

%%%%%%%%%%%%%%%%%%%%%%%%%%%%%%%%%%%%%%%%%%%%%%%%%%%%%%%%%%%%%%%%%%%%%%%%%%%
%%                               Variables                               %%
%%%%%%%%%%%%%%%%%%%%%%%%%%%%%%%%%%%%%%%%%%%%%%%%%%%%%%%%%%%%%%%%%%%%%%%%%%%

  \subsection{Variables}

    \vspace{-1cm}
\hspace{\varindent}\begin{longtable}{|p{\varnamewidth}|p{\vardescrwidth}|l}
\cline{1-2}
\cline{1-2} \centering \textbf{Name} & \centering \textbf{Description}& \\
\cline{1-2}
\endhead\cline{1-2}\multicolumn{3}{r}{\small\textit{continued on next page}}\\\endfoot\cline{1-2}
\endlastfoot\raggedright \_\-\_\-p\-a\-c\-k\-a\-g\-e\-\_\-\_\- & \raggedright \textbf{Value:} 
{\tt \texttt{'}\texttt{FractalZE}\texttt{'}}&\\
\cline{1-2}
\end{longtable}


%%%%%%%%%%%%%%%%%%%%%%%%%%%%%%%%%%%%%%%%%%%%%%%%%%%%%%%%%%%%%%%%%%%%%%%%%%%
%%                           Class Description                           %%
%%%%%%%%%%%%%%%%%%%%%%%%%%%%%%%%%%%%%%%%%%%%%%%%%%%%%%%%%%%%%%%%%%%%%%%%%%%

    \index{FractalZE \textit{(package)}!FractalZE.pascal \textit{(module)}!FractalZE.pascal.Pascal \textit{(class)}|(}
\subsection{Class Pascal}

    \label{FractalZE:pascal:Pascal}
\begin{tabular}{cccccc}
% Line for FractalZE.fractal.Fractal, linespec=[False]
\multicolumn{2}{r}{\settowidth{\BCL}{FractalZE.fractal.Fractal}\multirow{2}{\BCL}{FractalZE.fractal.Fractal}}
&&
  \\\cline{3-3}
  &&\multicolumn{1}{c|}{}
&&
  \\
&&\multicolumn{2}{l}{\textbf{FractalZE.pascal.Pascal}}
\end{tabular}

Fractal del triángulo de Pascal.

\begin{enumerate}

\setlength{\parskip}{0.5ex}
  \item Cada fila tiene un boque mas que la anterior.

  \item La primer fila tiene un solo bloque que vale 1.

  \item Cada fila comienza y termina con un bloque que vale uno.

  \item Cada bloque contiene la suma de los valores de los dos bloques 
    superiores.

\end{enumerate}


%%%%%%%%%%%%%%%%%%%%%%%%%%%%%%%%%%%%%%%%%%%%%%%%%%%%%%%%%%%%%%%%%%%%%%%%%%%
%%                                Methods                                %%
%%%%%%%%%%%%%%%%%%%%%%%%%%%%%%%%%%%%%%%%%%%%%%%%%%%%%%%%%%%%%%%%%%%%%%%%%%%

  \subsubsection{Methods}

    \label{FractalZE:pascal:Pascal:__init__}
    \index{FractalZE \textit{(package)}!FractalZE.pascal \textit{(module)}!FractalZE.pascal.Pascal \textit{(class)}!FractalZE.pascal.Pascal.\_\_init\_\_ \textit{(method)}}

    \vspace{0.5ex}

\hspace{.8\funcindent}\begin{boxedminipage}{\funcwidth}

    \raggedright \textbf{\_\_init\_\_}(\textit{self}, \textit{blockWidth}={\tt 10.0}, \textit{blockHeight}={\tt 10.0})

    \vspace{-1.5ex}

    \rule{\textwidth}{0.5\fboxrule}
\setlength{\parskip}{2ex}
\setlength{\parskip}{1ex}
      \textbf{Parameters}
      \vspace{-1ex}

      \begin{quote}
        \begin{Ventry}{xxxxxxxxxxx}

          \item[blockWidth]

          ancho del bloque.

            {\it (type=float)}

          \item[blockHeight]

          alto del bloque

            {\it (type=float)}

        \end{Ventry}

      \end{quote}

    \end{boxedminipage}

    \label{FractalZE:pascal:Pascal:getWidth}
    \index{FractalZE \textit{(package)}!FractalZE.pascal \textit{(module)}!FractalZE.pascal.Pascal \textit{(class)}!FractalZE.pascal.Pascal.getWidth \textit{(method)}}

    \vspace{0.5ex}

\hspace{.8\funcindent}\begin{boxedminipage}{\funcwidth}

    \raggedright \textbf{getWidth}(\textit{self}, \textit{n}={\tt 0})

    \vspace{-1.5ex}

    \rule{\textwidth}{0.5\fboxrule}
\setlength{\parskip}{2ex}
    Devuelve el ancho del fractal.

\setlength{\parskip}{1ex}
      \textbf{Return Value}
    \vspace{-1ex}

      \begin{quote}
      ancho del fractal

      {\it (type=float)}

      \end{quote}

    \end{boxedminipage}

    \label{FractalZE:pascal:Pascal:getHeight}
    \index{FractalZE \textit{(package)}!FractalZE.pascal \textit{(module)}!FractalZE.pascal.Pascal \textit{(class)}!FractalZE.pascal.Pascal.getHeight \textit{(method)}}

    \vspace{0.5ex}

\hspace{.8\funcindent}\begin{boxedminipage}{\funcwidth}

    \raggedright \textbf{getHeight}(\textit{self}, \textit{n}={\tt 0})

    \vspace{-1.5ex}

    \rule{\textwidth}{0.5\fboxrule}
\setlength{\parskip}{2ex}
    Devuelve la altura del fractal.

\setlength{\parskip}{1ex}
      \textbf{Return Value}
    \vspace{-1ex}

      \begin{quote}
      alto del fractal

      {\it (type=float)}

      \end{quote}

    \end{boxedminipage}


\large{\textbf{\textit{Inherited from FractalZE.fractal.Fractal\textit{(Section \ref{FractalZE:fractal:Fractal})}}}}

\begin{quote}
graph()
\end{quote}

%%%%%%%%%%%%%%%%%%%%%%%%%%%%%%%%%%%%%%%%%%%%%%%%%%%%%%%%%%%%%%%%%%%%%%%%%%%
%%                          Instance Variables                           %%
%%%%%%%%%%%%%%%%%%%%%%%%%%%%%%%%%%%%%%%%%%%%%%%%%%%%%%%%%%%%%%%%%%%%%%%%%%%

  \subsubsection{Instance Variables}

    \vspace{-1cm}
\hspace{\varindent}\begin{longtable}{|p{\varnamewidth}|p{\vardescrwidth}|l}
\cline{1-2}
\cline{1-2} \centering \textbf{Name} & \centering \textbf{Description}& \\
\cline{1-2}
\endhead\cline{1-2}\multicolumn{3}{r}{\small\textit{continued on next page}}\\\endfoot\cline{1-2}
\endlastfoot\raggedright f\-n\- & yellow&\\
\cline{1-2}
\end{longtable}

    \index{FractalZE \textit{(package)}!FractalZE.pascal \textit{(module)}!FractalZE.pascal.Pascal \textit{(class)}|)}
    \index{FractalZE \textit{(package)}!FractalZE.pascal \textit{(module)}|)}
