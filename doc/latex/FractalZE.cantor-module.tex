%
% API Documentation for API Documentation
% Module FractalZE.cantor
%
% Generated by epydoc 3.0.1
% [Fri Nov 28 18:59:36 2014]
%

%%%%%%%%%%%%%%%%%%%%%%%%%%%%%%%%%%%%%%%%%%%%%%%%%%%%%%%%%%%%%%%%%%%%%%%%%%%
%%                          Module Description                           %%
%%%%%%%%%%%%%%%%%%%%%%%%%%%%%%%%%%%%%%%%%%%%%%%%%%%%%%%%%%%%%%%%%%%%%%%%%%%

    \index{FractalZE \textit{(package)}!FractalZE.cantor \textit{(module)}|(}
\section{Module FractalZE.cantor}

    \label{FractalZE:cantor}

%%%%%%%%%%%%%%%%%%%%%%%%%%%%%%%%%%%%%%%%%%%%%%%%%%%%%%%%%%%%%%%%%%%%%%%%%%%
%%                               Variables                               %%
%%%%%%%%%%%%%%%%%%%%%%%%%%%%%%%%%%%%%%%%%%%%%%%%%%%%%%%%%%%%%%%%%%%%%%%%%%%

  \subsection{Variables}

    \vspace{-1cm}
\hspace{\varindent}\begin{longtable}{|p{\varnamewidth}|p{\vardescrwidth}|l}
\cline{1-2}
\cline{1-2} \centering \textbf{Name} & \centering \textbf{Description}& \\
\cline{1-2}
\endhead\cline{1-2}\multicolumn{3}{r}{\small\textit{continued on next page}}\\\endfoot\cline{1-2}
\endlastfoot\raggedright \_\-\_\-p\-a\-c\-k\-a\-g\-e\-\_\-\_\- & \raggedright \textbf{Value:} 
{\tt \texttt{'}\texttt{FractalZE}\texttt{'}}&\\
\cline{1-2}
\end{longtable}


%%%%%%%%%%%%%%%%%%%%%%%%%%%%%%%%%%%%%%%%%%%%%%%%%%%%%%%%%%%%%%%%%%%%%%%%%%%
%%                           Class Description                           %%
%%%%%%%%%%%%%%%%%%%%%%%%%%%%%%%%%%%%%%%%%%%%%%%%%%%%%%%%%%%%%%%%%%%%%%%%%%%

    \index{FractalZE \textit{(package)}!FractalZE.cantor \textit{(module)}!FractalZE.cantor.Cantor \textit{(class)}|(}
\subsection{Class Cantor}

    \label{FractalZE:cantor:Cantor}
\begin{tabular}{cccccc}
% Line for FractalZE.fractal.Fractal, linespec=[False]
\multicolumn{2}{r}{\settowidth{\BCL}{FractalZE.fractal.Fractal}\multirow{2}{\BCL}{FractalZE.fractal.Fractal}}
&&
  \\\cline{3-3}
  &&\multicolumn{1}{c|}{}
&&
  \\
&&\multicolumn{2}{l}{\textbf{FractalZE.cantor.Cantor}}
\end{tabular}

\textbf{Known Subclasses:} FractalZE.quintoCantor.QuintoCantor

Fractal de Cantor. Construccion:

\begin{enumerate}

\setlength{\parskip}{0.5ex}
  \item Se parte de un segmento.

  \item Se lo divide en 3 partes iguales.

  \item Se descarta la del medio.

  \item Se repite esto sucesivamente con cada segmento las veces que se desee.

\end{enumerate}


%%%%%%%%%%%%%%%%%%%%%%%%%%%%%%%%%%%%%%%%%%%%%%%%%%%%%%%%%%%%%%%%%%%%%%%%%%%
%%                                Methods                                %%
%%%%%%%%%%%%%%%%%%%%%%%%%%%%%%%%%%%%%%%%%%%%%%%%%%%%%%%%%%%%%%%%%%%%%%%%%%%

  \subsubsection{Methods}

    \label{FractalZE:cantor:Cantor:__init__}
    \index{FractalZE \textit{(package)}!FractalZE.cantor \textit{(module)}!FractalZE.cantor.Cantor \textit{(class)}!FractalZE.cantor.Cantor.\_\_init\_\_ \textit{(method)}}

    \vspace{0.5ex}

\hspace{.8\funcindent}\begin{boxedminipage}{\funcwidth}

    \raggedright \textbf{\_\_init\_\_}(\textit{self}, \textit{l}={\tt 1.0}, \textit{segmentHeight}={\tt 1})

    \vspace{-1.5ex}

    \rule{\textwidth}{0.5\fboxrule}
\setlength{\parskip}{2ex}
\setlength{\parskip}{1ex}
      \textbf{Parameters}
      \vspace{-1ex}

      \begin{quote}
        \begin{Ventry}{xxxxxxxxxxxxx}

          \item[l]

          longitud del segmento maximo. Default 1

            {\it (type=float)}

          \item[segmentHeight]

          altura que va a tener el segmento al dibujarlo(en pixels)

            {\it (type=int)}

        \end{Ventry}

      \end{quote}

    \end{boxedminipage}

    \label{FractalZE:cantor:Cantor:countSegments}
    \index{FractalZE \textit{(package)}!FractalZE.cantor \textit{(module)}!FractalZE.cantor.Cantor \textit{(class)}!FractalZE.cantor.Cantor.countSegments \textit{(method)}}

    \vspace{0.5ex}

\hspace{.8\funcindent}\begin{boxedminipage}{\funcwidth}

    \raggedright \textbf{countSegments}(\textit{self}, \textit{n})

    \vspace{-1.5ex}

    \rule{\textwidth}{0.5\fboxrule}
\setlength{\parskip}{2ex}
    Devuelve la cantidad de segmentos  del termino n.

\setlength{\parskip}{1ex}
      \textbf{Parameters}
      \vspace{-1ex}

      \begin{quote}
        \begin{Ventry}{x}

          \item[n]

          numero de termino de la sucesion

            {\it (type=int)}

        \end{Ventry}

      \end{quote}

      \textbf{Return Value}
    \vspace{-1ex}

      \begin{quote}
      cantidad de segmentos.

      {\it (type=int)}

      \end{quote}

    \end{boxedminipage}

    \label{FractalZE:cantor:Cantor:segmentLength}
    \index{FractalZE \textit{(package)}!FractalZE.cantor \textit{(module)}!FractalZE.cantor.Cantor \textit{(class)}!FractalZE.cantor.Cantor.segmentLength \textit{(method)}}

    \vspace{0.5ex}

\hspace{.8\funcindent}\begin{boxedminipage}{\funcwidth}

    \raggedright \textbf{segmentLength}(\textit{self}, \textit{n})

    \vspace{-1.5ex}

    \rule{\textwidth}{0.5\fboxrule}
\setlength{\parskip}{2ex}
    Devuelve la longitud de un segmento del termino n.

\setlength{\parskip}{1ex}
      \textbf{Parameters}
      \vspace{-1ex}

      \begin{quote}
        \begin{Ventry}{x}

          \item[n]

          numero de termino de la sucesion

            {\it (type=int)}

        \end{Ventry}

      \end{quote}

      \textbf{Return Value}
    \vspace{-1ex}

      \begin{quote}
      longitud de un segmento.

      {\it (type=float)}

      \end{quote}

    \end{boxedminipage}

    \label{FractalZE:cantor:Cantor:totalLength}
    \index{FractalZE \textit{(package)}!FractalZE.cantor \textit{(module)}!FractalZE.cantor.Cantor \textit{(class)}!FractalZE.cantor.Cantor.totalLength \textit{(method)}}

    \vspace{0.5ex}

\hspace{.8\funcindent}\begin{boxedminipage}{\funcwidth}

    \raggedright \textbf{totalLength}(\textit{self}, \textit{n})

    \vspace{-1.5ex}

    \rule{\textwidth}{0.5\fboxrule}
\setlength{\parskip}{2ex}
    Devuelve la suma de la longitud de todos los segmentos de un termino.

\setlength{\parskip}{1ex}
      \textbf{Parameters}
      \vspace{-1ex}

      \begin{quote}
        \begin{Ventry}{x}

          \item[n]

          numero de termino de la sucesion

            {\it (type=int)}

        \end{Ventry}

      \end{quote}

      \textbf{Return Value}
    \vspace{-1ex}

      \begin{quote}
      longitud total.

      {\it (type=float)}

      \end{quote}

    \end{boxedminipage}

    \label{FractalZE:cantor:Cantor:lindenmayer}
    \index{FractalZE \textit{(package)}!FractalZE.cantor \textit{(module)}!FractalZE.cantor.Cantor \textit{(class)}!FractalZE.cantor.Cantor.lindenmayer \textit{(method)}}

    \vspace{0.5ex}

\hspace{.8\funcindent}\begin{boxedminipage}{\funcwidth}

    \raggedright \textbf{lindenmayer}(\textit{self}, \textit{n}={\tt 0})

    \vspace{-1.5ex}

    \rule{\textwidth}{0.5\fboxrule}
\setlength{\parskip}{2ex}
    Devuelve el fractal de cantor expresado en L-System.

\setlength{\parskip}{1ex}
      \textbf{Parameters}
      \vspace{-1ex}

      \begin{quote}
        \begin{Ventry}{x}

          \item[n]

          termino de la sucesion

            {\it (type=int)}

        \end{Ventry}

      \end{quote}

      \textbf{Return Value}
    \vspace{-1ex}

      \begin{quote}
      termino expresado en L-system.

      {\it (type=string

      Algoritmo en L-System.

      \begin{itemize}
      \setlength{\parskip}{0.6ex}
        \item variables: \_ (space)

        \item constantes:

        \item axioma: \_

        \item reglas:

          \begin{enumerate}

          \setlength{\parskip}{0.5ex}
            \item \_ -{\textgreater} \_ \_

            \item (space) -{\textgreater} (space)(space)(space)

          \end{enumerate}

      \end{itemize}

      Interpretacion.

      \begin{enumerate}

      \setlength{\parskip}{0.5ex}
        \item Un \_(guion bajo) significa dibujar un segmento

        \item Un (space) significa avanzar la longitud de un segmento sin 
          dibujar

      \end{enumerate})}

      \end{quote}

    \end{boxedminipage}

    \label{FractalZE:cantor:Cantor:getWidth}
    \index{FractalZE \textit{(package)}!FractalZE.cantor \textit{(module)}!FractalZE.cantor.Cantor \textit{(class)}!FractalZE.cantor.Cantor.getWidth \textit{(method)}}

    \vspace{0.5ex}

\hspace{.8\funcindent}\begin{boxedminipage}{\funcwidth}

    \raggedright \textbf{getWidth}(\textit{self}, \textit{n}={\tt 0})

    \vspace{-1.5ex}

    \rule{\textwidth}{0.5\fboxrule}
\setlength{\parskip}{2ex}
    Devuelve el ancho del fractal.

\setlength{\parskip}{1ex}
      \textbf{Return Value}
    \vspace{-1ex}

      \begin{quote}
      ancho del fractal

      {\it (type=float)}

      \end{quote}

    \end{boxedminipage}

    \label{FractalZE:cantor:Cantor:getHeight}
    \index{FractalZE \textit{(package)}!FractalZE.cantor \textit{(module)}!FractalZE.cantor.Cantor \textit{(class)}!FractalZE.cantor.Cantor.getHeight \textit{(method)}}

    \vspace{0.5ex}

\hspace{.8\funcindent}\begin{boxedminipage}{\funcwidth}

    \raggedright \textbf{getHeight}(\textit{self}, \textit{n}={\tt 0})

    \vspace{-1.5ex}

    \rule{\textwidth}{0.5\fboxrule}
\setlength{\parskip}{2ex}
    Devuelve la altura del fractal.

\setlength{\parskip}{1ex}
      \textbf{Return Value}
    \vspace{-1ex}

      \begin{quote}
      alto del fractal

      {\it (type=float)}

      \end{quote}

    \end{boxedminipage}


\large{\textbf{\textit{Inherited from FractalZE.fractal.Fractal\textit{(Section \ref{FractalZE:fractal:Fractal})}}}}

\begin{quote}
graph()
\end{quote}
    \index{FractalZE \textit{(package)}!FractalZE.cantor \textit{(module)}!FractalZE.cantor.Cantor \textit{(class)}|)}
    \index{FractalZE \textit{(package)}!FractalZE.cantor \textit{(module)}|)}
