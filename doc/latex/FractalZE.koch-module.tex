%
% API Documentation for API Documentation
% Module FractalZE.koch
%
% Generated by epydoc 3.0.1
% [Fri Nov 28 18:59:36 2014]
%

%%%%%%%%%%%%%%%%%%%%%%%%%%%%%%%%%%%%%%%%%%%%%%%%%%%%%%%%%%%%%%%%%%%%%%%%%%%
%%                          Module Description                           %%
%%%%%%%%%%%%%%%%%%%%%%%%%%%%%%%%%%%%%%%%%%%%%%%%%%%%%%%%%%%%%%%%%%%%%%%%%%%

    \index{FractalZE \textit{(package)}!FractalZE.koch \textit{(module)}|(}
\section{Module FractalZE.koch}

    \label{FractalZE:koch}

%%%%%%%%%%%%%%%%%%%%%%%%%%%%%%%%%%%%%%%%%%%%%%%%%%%%%%%%%%%%%%%%%%%%%%%%%%%
%%                               Variables                               %%
%%%%%%%%%%%%%%%%%%%%%%%%%%%%%%%%%%%%%%%%%%%%%%%%%%%%%%%%%%%%%%%%%%%%%%%%%%%

  \subsection{Variables}

    \vspace{-1cm}
\hspace{\varindent}\begin{longtable}{|p{\varnamewidth}|p{\vardescrwidth}|l}
\cline{1-2}
\cline{1-2} \centering \textbf{Name} & \centering \textbf{Description}& \\
\cline{1-2}
\endhead\cline{1-2}\multicolumn{3}{r}{\small\textit{continued on next page}}\\\endfoot\cline{1-2}
\endlastfoot\raggedright \_\-\_\-p\-a\-c\-k\-a\-g\-e\-\_\-\_\- & \raggedright \textbf{Value:} 
{\tt \texttt{'}\texttt{FractalZE}\texttt{'}}&\\
\cline{1-2}
\end{longtable}


%%%%%%%%%%%%%%%%%%%%%%%%%%%%%%%%%%%%%%%%%%%%%%%%%%%%%%%%%%%%%%%%%%%%%%%%%%%
%%                           Class Description                           %%
%%%%%%%%%%%%%%%%%%%%%%%%%%%%%%%%%%%%%%%%%%%%%%%%%%%%%%%%%%%%%%%%%%%%%%%%%%%

    \index{FractalZE \textit{(package)}!FractalZE.koch \textit{(module)}!FractalZE.koch.Koch \textit{(class)}|(}
\subsection{Class Koch}

    \label{FractalZE:koch:Koch}
\begin{tabular}{cccccc}
% Line for FractalZE.fractal.Fractal, linespec=[False]
\multicolumn{2}{r}{\settowidth{\BCL}{FractalZE.fractal.Fractal}\multirow{2}{\BCL}{FractalZE.fractal.Fractal}}
&&
  \\\cline{3-3}
  &&\multicolumn{1}{c|}{}
&&
  \\
&&\multicolumn{2}{l}{\textbf{FractalZE.koch.Koch}}
\end{tabular}

\textbf{Known Subclasses:} FractalZE.snowflake.Snowflake

Fractal de la curva de Koch. Cosntruccion:

\begin{enumerate}

\setlength{\parskip}{0.5ex}
  \item Se parte de un segmento.

  \item Se divide el segmento en 3 partes iguales.

  \item Se construye un triangulo equilatero sobre el segmento central.

  \item Se descarta el segmento central(la base del triangulo).

  \item Se repiten los pasos para cada segmento las veces que se quiera.

\end{enumerate}


%%%%%%%%%%%%%%%%%%%%%%%%%%%%%%%%%%%%%%%%%%%%%%%%%%%%%%%%%%%%%%%%%%%%%%%%%%%
%%                                Methods                                %%
%%%%%%%%%%%%%%%%%%%%%%%%%%%%%%%%%%%%%%%%%%%%%%%%%%%%%%%%%%%%%%%%%%%%%%%%%%%

  \subsubsection{Methods}

    \label{FractalZE:koch:Koch:__init__}
    \index{FractalZE \textit{(package)}!FractalZE.koch \textit{(module)}!FractalZE.koch.Koch \textit{(class)}!FractalZE.koch.Koch.\_\_init\_\_ \textit{(method)}}

    \vspace{0.5ex}

\hspace{.8\funcindent}\begin{boxedminipage}{\funcwidth}

    \raggedright \textbf{\_\_init\_\_}(\textit{self}, \textit{l}={\tt 1})

    \vspace{-1.5ex}

    \rule{\textwidth}{0.5\fboxrule}
\setlength{\parskip}{2ex}
\setlength{\parskip}{1ex}
      \textbf{Parameters}
      \vspace{-1ex}

      \begin{quote}
        \begin{Ventry}{x}

          \item[l]

          longitud del segmento maximo(cuando n=0). Default 1

            {\it (type=float)}

        \end{Ventry}

      \end{quote}

    \end{boxedminipage}

    \label{FractalZE:koch:Koch:countSegments}
    \index{FractalZE \textit{(package)}!FractalZE.koch \textit{(module)}!FractalZE.koch.Koch \textit{(class)}!FractalZE.koch.Koch.countSegments \textit{(method)}}

    \vspace{0.5ex}

\hspace{.8\funcindent}\begin{boxedminipage}{\funcwidth}

    \raggedright \textbf{countSegments}(\textit{self}, \textit{n})

    \vspace{-1.5ex}

    \rule{\textwidth}{0.5\fboxrule}
\setlength{\parskip}{2ex}
    Devuelve la cantidad de segmentos del termino n.

\setlength{\parskip}{1ex}
      \textbf{Parameters}
      \vspace{-1ex}

      \begin{quote}
        \begin{Ventry}{x}

          \item[n]

          numero de termino de la sucesion

            {\it (type=int)}

        \end{Ventry}

      \end{quote}

      \textbf{Return Value}
    \vspace{-1ex}

      \begin{quote}
      cantidad de segmentos.

      {\it (type=int)}

      \end{quote}

    \end{boxedminipage}

    \label{FractalZE:koch:Koch:getHeight}
    \index{FractalZE \textit{(package)}!FractalZE.koch \textit{(module)}!FractalZE.koch.Koch \textit{(class)}!FractalZE.koch.Koch.getHeight \textit{(method)}}

    \vspace{0.5ex}

\hspace{.8\funcindent}\begin{boxedminipage}{\funcwidth}

    \raggedright \textbf{getHeight}(\textit{self}, \textit{n})

    \vspace{-1.5ex}

    \rule{\textwidth}{0.5\fboxrule}
\setlength{\parskip}{2ex}
    Devuelve la altura del fractal.

\setlength{\parskip}{1ex}
      \textbf{Return Value}
    \vspace{-1ex}

      \begin{quote}
      altura del fractal

      {\it (type=float)}

      \end{quote}

    \end{boxedminipage}

    \label{FractalZE:koch:Koch:getWidth}
    \index{FractalZE \textit{(package)}!FractalZE.koch \textit{(module)}!FractalZE.koch.Koch \textit{(class)}!FractalZE.koch.Koch.getWidth \textit{(method)}}

    \vspace{0.5ex}

\hspace{.8\funcindent}\begin{boxedminipage}{\funcwidth}

    \raggedright \textbf{getWidth}(\textit{self}, \textit{n}={\tt 0})

    \vspace{-1.5ex}

    \rule{\textwidth}{0.5\fboxrule}
\setlength{\parskip}{2ex}
    Devuelve el ancho del fractal.

\setlength{\parskip}{1ex}
      \textbf{Return Value}
    \vspace{-1ex}

      \begin{quote}
      ancho del fractal

      {\it (type=float)}

      \end{quote}

    \end{boxedminipage}

    \label{FractalZE:koch:Koch:lindenmayer}
    \index{FractalZE \textit{(package)}!FractalZE.koch \textit{(module)}!FractalZE.koch.Koch \textit{(class)}!FractalZE.koch.Koch.lindenmayer \textit{(method)}}

    \vspace{0.5ex}

\hspace{.8\funcindent}\begin{boxedminipage}{\funcwidth}

    \raggedright \textbf{lindenmayer}(\textit{self}, \textit{n})

    \vspace{-1.5ex}

    \rule{\textwidth}{0.5\fboxrule}
\setlength{\parskip}{2ex}
    Devuelve la curva de koch expresada en L-System

\setlength{\parskip}{1ex}
      \textbf{Parameters}
      \vspace{-1ex}

      \begin{quote}
        \begin{Ventry}{x}

          \item[n]

          numero de termino de la sucesion

            {\it (type=int)}

        \end{Ventry}

      \end{quote}

      \textbf{Return Value}
    \vspace{-1ex}

      \begin{quote}
      curva de koch expresada en L-System.

      {\it (type=string

      Algoritmo en L-System.

      \begin{itemize}
      \setlength{\parskip}{0.6ex}
        \item variables: \_

        \item constantes: +-

        \item axioma: \_

        \item reglas:

          \begin{enumerate}

          \setlength{\parskip}{0.5ex}
            \item \_ -{\textgreater} \_+\_-\_+\_

          \end{enumerate}

      \end{itemize}

      Interpretacion.

      \begin{enumerate}

      \setlength{\parskip}{0.5ex}
        \item Un \_(guion bajo) significa dibujar un segmento

        \item Un +(mas) significa girar en sentido antihorario 60 grados

      \end{enumerate}

      \begin{enumerate}

      \setlength{\parskip}{0.5ex}
        \item Un -(menos) significa girar en sentido horario 120 grados

      \end{enumerate})}

      \end{quote}

    \end{boxedminipage}

    \label{FractalZE:koch:Koch:segmentLength}
    \index{FractalZE \textit{(package)}!FractalZE.koch \textit{(module)}!FractalZE.koch.Koch \textit{(class)}!FractalZE.koch.Koch.segmentLength \textit{(method)}}

    \vspace{0.5ex}

\hspace{.8\funcindent}\begin{boxedminipage}{\funcwidth}

    \raggedright \textbf{segmentLength}(\textit{self}, \textit{n})

    \vspace{-1.5ex}

    \rule{\textwidth}{0.5\fboxrule}
\setlength{\parskip}{2ex}
    Devuelve la longitud de un segmento del termino n.

\setlength{\parskip}{1ex}
      \textbf{Parameters}
      \vspace{-1ex}

      \begin{quote}
        \begin{Ventry}{x}

          \item[n]

          numero de termino de la sucesion

            {\it (type=int)}

        \end{Ventry}

      \end{quote}

      \textbf{Return Value}
    \vspace{-1ex}

      \begin{quote}
      longitud de un segmento.

      {\it (type=float)}

      \end{quote}

    \end{boxedminipage}

    \label{FractalZE:koch:Koch:totalArea}
    \index{FractalZE \textit{(package)}!FractalZE.koch \textit{(module)}!FractalZE.koch.Koch \textit{(class)}!FractalZE.koch.Koch.totalArea \textit{(method)}}

    \vspace{0.5ex}

\hspace{.8\funcindent}\begin{boxedminipage}{\funcwidth}

    \raggedright \textbf{totalArea}(\textit{self}, \textit{n})

    \vspace{-1.5ex}

    \rule{\textwidth}{0.5\fboxrule}
\setlength{\parskip}{2ex}
    Devuelve el area total debajo de la curva.

\setlength{\parskip}{1ex}
      \textbf{Parameters}
      \vspace{-1ex}

      \begin{quote}
        \begin{Ventry}{x}

          \item[n]

          numero de termino de la sucesion

            {\it (type=int)}

        \end{Ventry}

      \end{quote}

      \textbf{Return Value}
    \vspace{-1ex}

      \begin{quote}
      area total debajo de la curva.

      {\it (type=float)}

      \end{quote}

    \end{boxedminipage}

    \label{FractalZE:koch:Koch:totalLength}
    \index{FractalZE \textit{(package)}!FractalZE.koch \textit{(module)}!FractalZE.koch.Koch \textit{(class)}!FractalZE.koch.Koch.totalLength \textit{(method)}}

    \vspace{0.5ex}

\hspace{.8\funcindent}\begin{boxedminipage}{\funcwidth}

    \raggedright \textbf{totalLength}(\textit{self}, \textit{n})

    \vspace{-1.5ex}

    \rule{\textwidth}{0.5\fboxrule}
\setlength{\parskip}{2ex}
    Devuelve la suma de la longitud de todos los segmentos.

\setlength{\parskip}{1ex}
      \textbf{Parameters}
      \vspace{-1ex}

      \begin{quote}
        \begin{Ventry}{x}

          \item[n]

          numero de termino de la sucesion

            {\it (type=int)}

        \end{Ventry}

      \end{quote}

      \textbf{Return Value}
    \vspace{-1ex}

      \begin{quote}
      longitud total.

      {\it (type=float)}

      \end{quote}

    \end{boxedminipage}


\large{\textbf{\textit{Inherited from FractalZE.fractal.Fractal\textit{(Section \ref{FractalZE:fractal:Fractal})}}}}

\begin{quote}
graph()
\end{quote}
    \index{FractalZE \textit{(package)}!FractalZE.koch \textit{(module)}!FractalZE.koch.Koch \textit{(class)}|)}
    \index{FractalZE \textit{(package)}!FractalZE.koch \textit{(module)}|)}
