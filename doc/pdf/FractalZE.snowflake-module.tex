%
% API Documentation for API Documentation
% Module FractalZE.snowflake
%
% Generated by epydoc 3.0.1
% [Fri Nov 28 18:59:14 2014]
%

%%%%%%%%%%%%%%%%%%%%%%%%%%%%%%%%%%%%%%%%%%%%%%%%%%%%%%%%%%%%%%%%%%%%%%%%%%%
%%                          Module Description                           %%
%%%%%%%%%%%%%%%%%%%%%%%%%%%%%%%%%%%%%%%%%%%%%%%%%%%%%%%%%%%%%%%%%%%%%%%%%%%

    \index{FractalZE \textit{(package)}!FractalZE.snowflake \textit{(module)}|(}
\section{Module FractalZE.snowflake}

    \label{FractalZE:snowflake}

%%%%%%%%%%%%%%%%%%%%%%%%%%%%%%%%%%%%%%%%%%%%%%%%%%%%%%%%%%%%%%%%%%%%%%%%%%%
%%                               Variables                               %%
%%%%%%%%%%%%%%%%%%%%%%%%%%%%%%%%%%%%%%%%%%%%%%%%%%%%%%%%%%%%%%%%%%%%%%%%%%%

  \subsection{Variables}

    \vspace{-1cm}
\hspace{\varindent}\begin{longtable}{|p{\varnamewidth}|p{\vardescrwidth}|l}
\cline{1-2}
\cline{1-2} \centering \textbf{Name} & \centering \textbf{Description}& \\
\cline{1-2}
\endhead\cline{1-2}\multicolumn{3}{r}{\small\textit{continued on next page}}\\\endfoot\cline{1-2}
\endlastfoot\raggedright \_\-\_\-p\-a\-c\-k\-a\-g\-e\-\_\-\_\- & \raggedright \textbf{Value:} 
{\tt \texttt{'}\texttt{FractalZE}\texttt{'}}&\\
\cline{1-2}
\end{longtable}


%%%%%%%%%%%%%%%%%%%%%%%%%%%%%%%%%%%%%%%%%%%%%%%%%%%%%%%%%%%%%%%%%%%%%%%%%%%
%%                           Class Description                           %%
%%%%%%%%%%%%%%%%%%%%%%%%%%%%%%%%%%%%%%%%%%%%%%%%%%%%%%%%%%%%%%%%%%%%%%%%%%%

    \index{FractalZE \textit{(package)}!FractalZE.snowflake \textit{(module)}!FractalZE.snowflake.Snowflake \textit{(class)}|(}
\subsection{Class Snowflake}

    \label{FractalZE:snowflake:Snowflake}
\begin{tabular}{cccccccc}
% Line for FractalZE.fractal.Fractal, linespec=[False, False]
\multicolumn{2}{r}{\settowidth{\BCL}{FractalZE.fractal.Fractal}\multirow{2}{\BCL}{FractalZE.fractal.Fractal}}
&&
&&
  \\\cline{3-3}
  &&\multicolumn{1}{c|}{}
&&
&&
  \\
% Line for FractalZE.koch.Koch, linespec=[False]
\multicolumn{4}{r}{\settowidth{\BCL}{FractalZE.koch.Koch}\multirow{2}{\BCL}{FractalZE.koch.Koch}}
&&
  \\\cline{5-5}
  &&&&\multicolumn{1}{c|}{}
&&
  \\
&&&&\multicolumn{2}{l}{\textbf{FractalZE.snowflake.Snowflake}}
\end{tabular}

Fractal de la bola de nieve. Cosntruccion:

\begin{enumerate}

\setlength{\parskip}{0.5ex}
  \item Se parte de un triangulo equilátero.

  \item Se divide cada segmento en 3 partes iguales.

  \item Se construye un triangulo equilatero sobre cada mitad de cada segmento.

  \item Se descartan los segmentos centrales(la base de los triangulo creados).

  \item Se repiten los pasos para cada segmento las veces que se quiera.

\end{enumerate}


%%%%%%%%%%%%%%%%%%%%%%%%%%%%%%%%%%%%%%%%%%%%%%%%%%%%%%%%%%%%%%%%%%%%%%%%%%%
%%                                Methods                                %%
%%%%%%%%%%%%%%%%%%%%%%%%%%%%%%%%%%%%%%%%%%%%%%%%%%%%%%%%%%%%%%%%%%%%%%%%%%%

  \subsubsection{Methods}

    \vspace{0.5ex}

\hspace{.8\funcindent}\begin{boxedminipage}{\funcwidth}

    \raggedright \textbf{countSegments}(\textit{self}, \textit{n})

    \vspace{-1.5ex}

    \rule{\textwidth}{0.5\fboxrule}
\setlength{\parskip}{2ex}
    Devuelve la cantidad de segmentos del termino n.

\setlength{\parskip}{1ex}
      \textbf{Parameters}
      \vspace{-1ex}

      \begin{quote}
        \begin{Ventry}{x}

          \item[n]

          numero de termino de la sucesion

            {\it (type=int)}

        \end{Ventry}

      \end{quote}

      \textbf{Return Value}
    \vspace{-1ex}

      \begin{quote}
      cantidad de segmentos.

      {\it (type=int)}

      \end{quote}

      Overrides: FractalZE.koch.Koch.countSegments

    \end{boxedminipage}

    \vspace{0.5ex}

\hspace{.8\funcindent}\begin{boxedminipage}{\funcwidth}

    \raggedright \textbf{totalArea}(\textit{self}, \textit{n})

    \vspace{-1.5ex}

    \rule{\textwidth}{0.5\fboxrule}
\setlength{\parskip}{2ex}
    Devuelve el area total debajo de la curva.

\setlength{\parskip}{1ex}
      \textbf{Parameters}
      \vspace{-1ex}

      \begin{quote}
        \begin{Ventry}{x}

          \item[n]

          numero de termino de la sucesion

            {\it (type=int)}

        \end{Ventry}

      \end{quote}

      \textbf{Return Value}
    \vspace{-1ex}

      \begin{quote}
      area total debajo de la curva.

      {\it (type=float)}

      \end{quote}

      Overrides: FractalZE.koch.Koch.totalArea

    \end{boxedminipage}

    \vspace{0.5ex}

\hspace{.8\funcindent}\begin{boxedminipage}{\funcwidth}

    \raggedright \textbf{getWidth}(\textit{self}, \textit{n}={\tt 0})

    \vspace{-1.5ex}

    \rule{\textwidth}{0.5\fboxrule}
\setlength{\parskip}{2ex}
    Devuelve el ancho del fractal.

\setlength{\parskip}{1ex}
      \textbf{Return Value}
    \vspace{-1ex}

      \begin{quote}
      ancho del fractal

      {\it (type=float)}

      \end{quote}

      Overrides: FractalZE.koch.Koch.getWidth

    \end{boxedminipage}

    \vspace{0.5ex}

\hspace{.8\funcindent}\begin{boxedminipage}{\funcwidth}

    \raggedright \textbf{getHeight}(\textit{self}, \textit{n}={\tt 0})

    \vspace{-1.5ex}

    \rule{\textwidth}{0.5\fboxrule}
\setlength{\parskip}{2ex}
    Devuelve la altura del fractal.

\setlength{\parskip}{1ex}
      \textbf{Return Value}
    \vspace{-1ex}

      \begin{quote}
      altura del fractal

      {\it (type=float)}

      \end{quote}

      Overrides: FractalZE.koch.Koch.getHeight

    \end{boxedminipage}


\large{\textbf{\textit{Inherited from FractalZE.koch.Koch\textit{(Section \ref{FractalZE:koch:Koch})}}}}

\begin{quote}
\_\_init\_\_(), lindenmayer(), segmentLength(), totalLength()
\end{quote}

\large{\textbf{\textit{Inherited from FractalZE.fractal.Fractal\textit{(Section \ref{FractalZE:fractal:Fractal})}}}}

\begin{quote}
graph()
\end{quote}
    \index{FractalZE \textit{(package)}!FractalZE.snowflake \textit{(module)}!FractalZE.snowflake.Snowflake \textit{(class)}|)}
    \index{FractalZE \textit{(package)}!FractalZE.snowflake \textit{(module)}|)}
