%
% API Documentation for API Documentation
% Module FractalZE.quintoCantor
%
% Generated by epydoc 3.0.1
% [Fri Nov 28 18:59:14 2014]
%

%%%%%%%%%%%%%%%%%%%%%%%%%%%%%%%%%%%%%%%%%%%%%%%%%%%%%%%%%%%%%%%%%%%%%%%%%%%
%%                          Module Description                           %%
%%%%%%%%%%%%%%%%%%%%%%%%%%%%%%%%%%%%%%%%%%%%%%%%%%%%%%%%%%%%%%%%%%%%%%%%%%%

    \index{FractalZE \textit{(package)}!FractalZE.quintoCantor \textit{(module)}|(}
\section{Module FractalZE.quintoCantor}

    \label{FractalZE:quintoCantor}

%%%%%%%%%%%%%%%%%%%%%%%%%%%%%%%%%%%%%%%%%%%%%%%%%%%%%%%%%%%%%%%%%%%%%%%%%%%
%%                               Variables                               %%
%%%%%%%%%%%%%%%%%%%%%%%%%%%%%%%%%%%%%%%%%%%%%%%%%%%%%%%%%%%%%%%%%%%%%%%%%%%

  \subsection{Variables}

    \vspace{-1cm}
\hspace{\varindent}\begin{longtable}{|p{\varnamewidth}|p{\vardescrwidth}|l}
\cline{1-2}
\cline{1-2} \centering \textbf{Name} & \centering \textbf{Description}& \\
\cline{1-2}
\endhead\cline{1-2}\multicolumn{3}{r}{\small\textit{continued on next page}}\\\endfoot\cline{1-2}
\endlastfoot\raggedright \_\-\_\-p\-a\-c\-k\-a\-g\-e\-\_\-\_\- & \raggedright \textbf{Value:} 
{\tt \texttt{'}\texttt{FractalZE}\texttt{'}}&\\
\cline{1-2}
\end{longtable}


%%%%%%%%%%%%%%%%%%%%%%%%%%%%%%%%%%%%%%%%%%%%%%%%%%%%%%%%%%%%%%%%%%%%%%%%%%%
%%                           Class Description                           %%
%%%%%%%%%%%%%%%%%%%%%%%%%%%%%%%%%%%%%%%%%%%%%%%%%%%%%%%%%%%%%%%%%%%%%%%%%%%

    \index{FractalZE \textit{(package)}!FractalZE.quintoCantor \textit{(module)}!FractalZE.quintoCantor.QuintoCantor \textit{(class)}|(}
\subsection{Class QuintoCantor}

    \label{FractalZE:quintoCantor:QuintoCantor}
\begin{tabular}{cccccccc}
% Line for FractalZE.fractal.Fractal, linespec=[False, False]
\multicolumn{2}{r}{\settowidth{\BCL}{FractalZE.fractal.Fractal}\multirow{2}{\BCL}{FractalZE.fractal.Fractal}}
&&
&&
  \\\cline{3-3}
  &&\multicolumn{1}{c|}{}
&&
&&
  \\
% Line for FractalZE.cantor.Cantor, linespec=[False]
\multicolumn{4}{r}{\settowidth{\BCL}{FractalZE.cantor.Cantor}\multirow{2}{\BCL}{FractalZE.cantor.Cantor}}
&&
  \\\cline{5-5}
  &&&&\multicolumn{1}{c|}{}
&&
  \\
&&&&\multicolumn{2}{l}{\textbf{FractalZE.quintoCantor.QuintoCantor}}
\end{tabular}

Fractal de cantor que divide los segmentos en 5 y descarta el del medio.


%%%%%%%%%%%%%%%%%%%%%%%%%%%%%%%%%%%%%%%%%%%%%%%%%%%%%%%%%%%%%%%%%%%%%%%%%%%
%%                                Methods                                %%
%%%%%%%%%%%%%%%%%%%%%%%%%%%%%%%%%%%%%%%%%%%%%%%%%%%%%%%%%%%%%%%%%%%%%%%%%%%

  \subsubsection{Methods}

    \vspace{0.5ex}

\hspace{.8\funcindent}\begin{boxedminipage}{\funcwidth}

    \raggedright \textbf{segmentLength}(\textit{self}, \textit{n})

    \vspace{-1.5ex}

    \rule{\textwidth}{0.5\fboxrule}
\setlength{\parskip}{2ex}
    Devuelve la longitud de un segmento del termino n.

\setlength{\parskip}{1ex}
      \textbf{Parameters}
      \vspace{-1ex}

      \begin{quote}
        \begin{Ventry}{x}

          \item[n]

          numero de termino de la sucesion

            {\it (type=int)}

        \end{Ventry}

      \end{quote}

      \textbf{Return Value}
    \vspace{-1ex}

      \begin{quote}
      longitud de un segmento.

      {\it (type=float)}

      \end{quote}

      Overrides: FractalZE.cantor.Cantor.segmentLength

    \end{boxedminipage}

    \vspace{0.5ex}

\hspace{.8\funcindent}\begin{boxedminipage}{\funcwidth}

    \raggedright \textbf{lindenmayer}(\textit{self})

\setlength{\parskip}{2ex}
    Devuelve el fractal de cantor expresado en L-System.

\setlength{\parskip}{1ex}
      \textbf{Parameters}
      \vspace{-1ex}

      \begin{quote}
        \begin{Ventry}{x}

          \item[n]

          termino de la sucesion

        \end{Ventry}

      \end{quote}

      \textbf{Return Value}
    \vspace{-1ex}

      \begin{quote}
      termino expresado en L-system.

      {\it (type=string

      Algoritmo en L-System.

      \begin{itemize}
      \setlength{\parskip}{0.6ex}
        \item variables: \_ (space)

        \item constantes:

        \item axioma: \_

        \item reglas:

          \begin{enumerate}

          \setlength{\parskip}{0.5ex}
            \item \_ -{\textgreater} \_ \_

            \item (space) -{\textgreater} (space)(space)(space)

          \end{enumerate}

      \end{itemize}

      Interpretacion.

      \begin{enumerate}

      \setlength{\parskip}{0.5ex}
        \item Un \_(guion bajo) significa dibujar un segmento

        \item Un (space) significa avanzar la longitud de un segmento sin 
          dibujar

      \end{enumerate})}

      \end{quote}

      Overrides: FractalZE.cantor.Cantor.lindenmayer 	extit{(inherited documentation)}

    \end{boxedminipage}


\large{\textbf{\textit{Inherited from FractalZE.cantor.Cantor\textit{(Section \ref{FractalZE:cantor:Cantor})}}}}

\begin{quote}
\_\_init\_\_(), countSegments(), getHeight(), getWidth(), totalLength()
\end{quote}

\large{\textbf{\textit{Inherited from FractalZE.fractal.Fractal\textit{(Section \ref{FractalZE:fractal:Fractal})}}}}

\begin{quote}
graph()
\end{quote}
    \index{FractalZE \textit{(package)}!FractalZE.quintoCantor \textit{(module)}!FractalZE.quintoCantor.QuintoCantor \textit{(class)}|)}
    \index{FractalZE \textit{(package)}!FractalZE.quintoCantor \textit{(module)}|)}
