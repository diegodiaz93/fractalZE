%
% API Documentation for fractalZE
% Module FractalZE.arbol
%
% Generated by epydoc 3.0.1
% [Mon Jun  8 00:25:39 2015]
%

%%%%%%%%%%%%%%%%%%%%%%%%%%%%%%%%%%%%%%%%%%%%%%%%%%%%%%%%%%%%%%%%%%%%%%%%%%%
%%                          Module Description                           %%
%%%%%%%%%%%%%%%%%%%%%%%%%%%%%%%%%%%%%%%%%%%%%%%%%%%%%%%%%%%%%%%%%%%%%%%%%%%

    \index{FractalZE \textit{(package)}!FractalZE.arbol \textit{(module)}|(}
\section{Module FractalZE.arbol}

    \label{FractalZE:arbol}

%%%%%%%%%%%%%%%%%%%%%%%%%%%%%%%%%%%%%%%%%%%%%%%%%%%%%%%%%%%%%%%%%%%%%%%%%%%
%%                               Variables                               %%
%%%%%%%%%%%%%%%%%%%%%%%%%%%%%%%%%%%%%%%%%%%%%%%%%%%%%%%%%%%%%%%%%%%%%%%%%%%

  \subsection{Variables}

    \vspace{-1cm}
\hspace{\varindent}\begin{longtable}{|p{\varnamewidth}|p{\vardescrwidth}|l}
\cline{1-2}
\cline{1-2} \centering \textbf{Name} & \centering \textbf{Description}& \\
\cline{1-2}
\endhead\cline{1-2}\multicolumn{3}{r}{\small\textit{continued on next page}}\\\endfoot\cline{1-2}
\endlastfoot\raggedright \_\-\_\-p\-a\-c\-k\-a\-g\-e\-\_\-\_\- & \raggedright \textbf{Value:} 
{\tt \texttt{'}\texttt{FractalZE}\texttt{'}}&\\
\cline{1-2}
\end{longtable}


%%%%%%%%%%%%%%%%%%%%%%%%%%%%%%%%%%%%%%%%%%%%%%%%%%%%%%%%%%%%%%%%%%%%%%%%%%%
%%                           Class Description                           %%
%%%%%%%%%%%%%%%%%%%%%%%%%%%%%%%%%%%%%%%%%%%%%%%%%%%%%%%%%%%%%%%%%%%%%%%%%%%

    \index{FractalZE \textit{(package)}!FractalZE.arbol \textit{(module)}!FractalZE.arbol.Arbol \textit{(class)}|(}
\subsection{Class Arbol}

    \label{FractalZE:arbol:Arbol}
\begin{tabular}{cccccc}
% Line for FractalZE.fractal.Fractal, linespec=[False]
\multicolumn{2}{r}{\settowidth{\BCL}{FractalZE.fractal.Fractal}\multirow{2}{\BCL}{FractalZE.fractal.Fractal}}
&&
  \\\cline{3-3}
  &&\multicolumn{1}{c|}{}
&&
  \\
&&\multicolumn{2}{l}{\textbf{FractalZE.arbol.Arbol}}
\end{tabular}

Fractal del arbol. Construccion:

\begin{enumerate}

\setlength{\parskip}{0.5ex}
  \item Se parte de un segmento.

  \item Desde la punta de cada rama se dibujan dos nuevos segmentos de la mitad
    de la longitud de la rama.

    \begin{enumerate}

    \setlength{\parskip}{0.5ex}
      \item El primero apuntando 60 grados a la izquierda de la direccion de la
        rama.

      \item El segundo apuntando 60 grados a la derecha de la direccion de la 
        rama.

    \end{enumerate}

  \item Se repite esto sucesivamente con cada rama las veces que se desee.

\end{enumerate}


%%%%%%%%%%%%%%%%%%%%%%%%%%%%%%%%%%%%%%%%%%%%%%%%%%%%%%%%%%%%%%%%%%%%%%%%%%%
%%                                Methods                                %%
%%%%%%%%%%%%%%%%%%%%%%%%%%%%%%%%%%%%%%%%%%%%%%%%%%%%%%%%%%%%%%%%%%%%%%%%%%%

  \subsubsection{Methods}

    \label{FractalZE:arbol:Arbol:__init__}
    \index{FractalZE \textit{(package)}!FractalZE.arbol \textit{(module)}!FractalZE.arbol.Arbol \textit{(class)}!FractalZE.arbol.Arbol.\_\_init\_\_ \textit{(method)}}

    \vspace{0.5ex}

\hspace{.8\funcindent}\begin{boxedminipage}{\funcwidth}

    \raggedright \textbf{\_\_init\_\_}(\textit{self}, \textit{l}={\tt 1}, \textit{angle}={\tt 0.523598775598})

    \vspace{-1.5ex}

    \rule{\textwidth}{0.5\fboxrule}
\setlength{\parskip}{2ex}
\setlength{\parskip}{1ex}
      \textbf{Parameters}
      \vspace{-1ex}

      \begin{quote}
        \begin{Ventry}{xxxxx}

          \item[l]

          longitud del segmento maximo.

            {\it (type=float)}

          \item[angle]

          angulo ramas en radianes

            {\it (type=float)}

        \end{Ventry}

      \end{quote}

    \end{boxedminipage}

    \label{FractalZE:arbol:Arbol:segmentLength}
    \index{FractalZE \textit{(package)}!FractalZE.arbol \textit{(module)}!FractalZE.arbol.Arbol \textit{(class)}!FractalZE.arbol.Arbol.segmentLength \textit{(method)}}

    \vspace{0.5ex}

\hspace{.8\funcindent}\begin{boxedminipage}{\funcwidth}

    \raggedright \textbf{segmentLength}(\textit{self}, \textit{n}={\tt 0})

    \vspace{-1.5ex}

    \rule{\textwidth}{0.5\fboxrule}
\setlength{\parskip}{2ex}
    Devuelve la longitud de un segmento del termino n.

\setlength{\parskip}{1ex}
      \textbf{Parameters}
      \vspace{-1ex}

      \begin{quote}
        \begin{Ventry}{x}

          \item[n]

          numero de termino de la sucesion

            {\it (type=int)}

        \end{Ventry}

      \end{quote}

      \textbf{Return Value}
    \vspace{-1ex}

      \begin{quote}
      longitud de un segmento.

      {\it (type=float)}

      \end{quote}

    \end{boxedminipage}

    \label{FractalZE:arbol:Arbol:countSegments}
    \index{FractalZE \textit{(package)}!FractalZE.arbol \textit{(module)}!FractalZE.arbol.Arbol \textit{(class)}!FractalZE.arbol.Arbol.countSegments \textit{(method)}}

    \vspace{0.5ex}

\hspace{.8\funcindent}\begin{boxedminipage}{\funcwidth}

    \raggedright \textbf{countSegments}(\textit{self}, \textit{n})

    \vspace{-1.5ex}

    \rule{\textwidth}{0.5\fboxrule}
\setlength{\parskip}{2ex}
    Devuelve la cantidad de segmentos nuevos.

\setlength{\parskip}{1ex}
      \textbf{Parameters}
      \vspace{-1ex}

      \begin{quote}
        \begin{Ventry}{x}

          \item[n]

          numero de termino de la sucesion

            {\it (type=int)}

        \end{Ventry}

      \end{quote}

      \textbf{Return Value}
    \vspace{-1ex}

      \begin{quote}
      cantidad de segmentos.

      {\it (type=int)}

      \end{quote}

    \end{boxedminipage}

    \label{FractalZE:arbol:Arbol:totalLength}
    \index{FractalZE \textit{(package)}!FractalZE.arbol \textit{(module)}!FractalZE.arbol.Arbol \textit{(class)}!FractalZE.arbol.Arbol.totalLength \textit{(method)}}

    \vspace{0.5ex}

\hspace{.8\funcindent}\begin{boxedminipage}{\funcwidth}

    \raggedright \textbf{totalLength}(\textit{self}, \textit{n})

    \vspace{-1.5ex}

    \rule{\textwidth}{0.5\fboxrule}
\setlength{\parskip}{2ex}
    Devuelve la suma de la longitud de todos los segmentos.

\setlength{\parskip}{1ex}
      \textbf{Parameters}
      \vspace{-1ex}

      \begin{quote}
        \begin{Ventry}{x}

          \item[n]

          numero de termino de la sucesion

            {\it (type=int)}

        \end{Ventry}

      \end{quote}

      \textbf{Return Value}
    \vspace{-1ex}

      \begin{quote}
      longitud total.

      {\it (type=float)}

      \end{quote}

    \end{boxedminipage}

    \label{FractalZE:arbol:Arbol:lindenmayer}
    \index{FractalZE \textit{(package)}!FractalZE.arbol \textit{(module)}!FractalZE.arbol.Arbol \textit{(class)}!FractalZE.arbol.Arbol.lindenmayer \textit{(method)}}

    \vspace{0.5ex}

\hspace{.8\funcindent}\begin{boxedminipage}{\funcwidth}

    \raggedright \textbf{lindenmayer}(\textit{self}, \textit{n})

    \vspace{-1.5ex}

    \rule{\textwidth}{0.5\fboxrule}
\setlength{\parskip}{2ex}
    Devuelve el fractal del arbol expresado en L-System

\setlength{\parskip}{1ex}
      \textbf{Parameters}
      \vspace{-1ex}

      \begin{quote}
        \begin{Ventry}{x}

          \item[n]

          numero de termino de la sucesion

            {\it (type=int)}

        \end{Ventry}

      \end{quote}

      \textbf{Return Value}
    \vspace{-1ex}

      \begin{quote}
      fractal expresado en L-System.

      {\it (type=string

      Algoritmo en L-System.

      \begin{itemize}
      \setlength{\parskip}{0.6ex}
        \item variables: 0 1

        \item constantes: [ ]

        \item axioma: 0

        \item reglas:

          \begin{enumerate}

          \setlength{\parskip}{0.5ex}
            \item 1 -{\textgreater} 11

            \item 0 -{\textgreater} 1[0]0

          \end{enumerate}

      \end{itemize}

      Interpretacion.

      \begin{enumerate}

      \setlength{\parskip}{0.5ex}
        \item Un 0 significa dibujar una hoja (segmento de longitud L/2)

        \item Un 1 significa dibujar una rama (segmento de longitud L)

        \item Un [ significa guardar posicion y angulo en la pila, y girar X 
          grados en sentido antihorario

        \item Un ] significa extraer posicion y angulo de la pila, y girar X 
          grados en sentido horario

      \end{enumerate})}

      \end{quote}

    \end{boxedminipage}

    \label{FractalZE:arbol:Arbol:getHeight}
    \index{FractalZE \textit{(package)}!FractalZE.arbol \textit{(module)}!FractalZE.arbol.Arbol \textit{(class)}!FractalZE.arbol.Arbol.getHeight \textit{(method)}}

    \vspace{0.5ex}

\hspace{.8\funcindent}\begin{boxedminipage}{\funcwidth}

    \raggedright \textbf{getHeight}(\textit{self}, \textit{n}, \textit{angle}={\tt 1.57079632679})

    \vspace{-1.5ex}

    \rule{\textwidth}{0.5\fboxrule}
\setlength{\parskip}{2ex}
    Devuelve la altura del arbol.

\setlength{\parskip}{1ex}
      \textbf{Parameters}
      \vspace{-1ex}

      \begin{quote}
        \begin{Ventry}{xxxxx}

          \item[n]

          numero de termino de la sucesion

            {\it (type=int)}

          \item[angle]

          no se usa, necesario para la recursion

            {\it (type=float)}

        \end{Ventry}

      \end{quote}

      \textbf{Return Value}
    \vspace{-1ex}

      \begin{quote}
      altura del arbol.

      {\it (type=float)}

      \end{quote}

    \end{boxedminipage}

    \label{FractalZE:arbol:Arbol:getWidth}
    \index{FractalZE \textit{(package)}!FractalZE.arbol \textit{(module)}!FractalZE.arbol.Arbol \textit{(class)}!FractalZE.arbol.Arbol.getWidth \textit{(method)}}

    \vspace{0.5ex}

\hspace{.8\funcindent}\begin{boxedminipage}{\funcwidth}

    \raggedright \textbf{getWidth}(\textit{self}, \textit{n}={\tt 0}, \textit{angle}={\tt 1.57079632679})

    \vspace{-1.5ex}

    \rule{\textwidth}{0.5\fboxrule}
\setlength{\parskip}{2ex}
    Devuelve el ancho del arbol.

\setlength{\parskip}{1ex}
      \textbf{Parameters}
      \vspace{-1ex}

      \begin{quote}
        \begin{Ventry}{xxxxx}

          \item[n]

          numero de termino de la sucesion

            {\it (type=int)}

          \item[angle]

          no se usa, necesario para la recursion

            {\it (type=float)}

        \end{Ventry}

      \end{quote}

      \textbf{Return Value}
    \vspace{-1ex}

      \begin{quote}
      ancho del arbol.

      {\it (type=float)}

      \end{quote}

    \end{boxedminipage}


\large{\textbf{\textit{Inherited from FractalZE.fractal.Fractal\textit{(Section \ref{FractalZE:fractal:Fractal})}}}}

\begin{quote}
graph()
\end{quote}
    \index{FractalZE \textit{(package)}!FractalZE.arbol \textit{(module)}!FractalZE.arbol.Arbol \textit{(class)}|)}
    \index{FractalZE \textit{(package)}!FractalZE.arbol \textit{(module)}|)}
